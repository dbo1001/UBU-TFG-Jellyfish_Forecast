\apendice{Documentación técnica de programación}

\section{Introducción}
En este apéndice se explicarán todos los aspectos relevantes del proyecto con el objetivo de facilitar la comprensión del mismo si otro desarrollador quisiera continuar con el trabajo o entenderlo más a fondo.

\section{Estructura de directorios}

El proyecto está alojado en dos repositorios diferentes. El motivo, es el de conseguir un despliegue automático a través de la plataforma \emph{heroku}. Por lo que nos encontramos con:

\subsection{Web Jellyfish Forecast}
Es el repositorio\footnote{Web Jellyfish Forecast. \url{https://github.com/psnti/WebJellyfishForecast}} en el que se alojan todos los ficheros necesarios para el funcionamiento de la aplicación web.

Contiene los siguientes directorios:
\begin{itemize}
	\item \textbf{Documentos}: Se guardan ,el dataframe en el que estan la relación de playas con sus coordenadas así como un fichero excel con los avistamientos, necesario para realizar el historial de avistamientos.
	\item \textbf{static}: Aloja los ficheros css y javascript así como las imágenes necesarios para el buen funcionamiento de la aplicación.
	\item \textbf{templates}: Contiene los ficheros html.
\end{itemize}

Tambien contiene los siguientes archivos:
\begin{itemize}
	\item \textbf{Procfile}: Archivo necesario para el despliegue en \emph{Heroku} que contiene el comando que debe ejecutar la app en el arranque.
	\item \textbf{app.py}: Archivo python con la funcionalidad de la aplicación.
	\item \textbf{requirements.txt}: Listado de bibliotecas que se deben instalar en la máquina antes de ejecutar la aplicación.
\end{itemize}

\subsection{Jellyfish Forecast}
Es el repositorio\footnote{Jellyfish Forecast. \url{https://github.com/psnti/Jellyfish_Forecast}} en el que se alojan el resto de archivos del proyecto.

Contiene los siguientes directorios:
\begin{itemize}
	\item \textbf{Web}: Contiene el boceto de la aplicación realizado antes del desarrollo de la misma asi como una referencia al anterior repositorio.
	\item \textbf{docs/latex}: Aloja la documentación del proyecto.
	\item \textbf{src}: Contiene todo el código utilizado para la descarga y tratamiento de los datos así como el entrenamiento del modelo.
\end{itemize}

\section{Manual del programador}

Este apartado está destinado a la explicación de la instalación de todo lo necesario para poder ejecutar el proyecto o programadores que quieran trabajar en la mejora del mismo.





\section{Compilación, instalación y ejecución del proyecto}





\section{Pruebas del sistema}


