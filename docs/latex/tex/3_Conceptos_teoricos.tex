\capitulo{3}{Conceptos teóricos}

%En aquellos proyectos que necesiten para su comprensión y desarrollo de unos conceptos teóricos de una determinada materia o de un determinado dominio de conocimiento, debe existir un apartado que sintetice dichos conceptos.

\section{Medusas}
Las medusas son animales marinos formados por un cuerpo gelatinoso del que cuelga un manubrio tubular, encontrando la boca en la parte inferior de este. Algunas especies, tienen tentáculos con celular urticantes denominadas cnidocitos. Las medusas se desplazan mediante contracciones de su cuerpo absorbiendo agua para luego ser expulsada de manera brusca provocando el movimiento~\cite{wiki:medusas}.

Su reproducción es asexual, siendo esta, una fecundación externa mediante óvulos y espermatozoides que son liberados por los machos y las hembras respectivamente. Esto da lugar a la fecundación de gametos que se convertirán en larvas denominadas plánulas. Más adelante estas larvas se adhieren a alguna superficie donde se transformarán en pólipos para finalmente desprenderse la medusa adulta~\cite{noauthor_reproduccion_2016}.

\imagenflotante{reproducciónMedusas.jpg}{Fases del proceso de reproducción de las medusas.\cite{reproduccionMedusas}}

Su alimentación se basa principalmente en plancton aunque también son capaces de comer crustáceos, huevos o peces pequeños.

\subsection{Comportamiento de las medusas}
Los colonias de medusas están muy influenciadas por las condiciones climatológicas y marítimas. La temperatura, salinidad, el viento y las corrientes son los principales factores a tener en cuenta.

El peso de estos factores en los desplazamientos de las colonias varía en función de la etapa de desarrollo en la que se encuentren. Las que tienen un tamaño pequeño o mediano, están más condicionadas a la dirección de los vientos y las corrientes ya que, debido a su pequeño tamaño, no son capaces de contrarrestar estas fuerzas. Por otro lado, en las medusas de un tamaño superior, estos factores tienen menos relevancia mientras que la salinidad del agua y la temperatura de la misma adquieren un mayor protagonismo. Hasta unos 25 grados el numero de medusas va en aumento. A partir de ese punto, la concentración de las mismas decrece.

\say{La reproducción de estas también se ve influenciada por la temperatura del agua pues según diferentes experimentos se ha demostrado que existe una relación entre el aumento de las temperaturas y una mayor reproducción asexual en varias especies gelatinosas.} \cite{canepa_environmental_2017}

%\say{prueba de cita literal}

Los factores humanos también tienen su influencia en los movimientos de las poblaciones de medusas y en su reproducción. 
El aumento de materia orgánica provocado por vertidos como podrían ser los de una EDAR (Estación Depuradora de Aguas Residuales) o de una explotación agrícola, que provoca la eutrofización del medio haciendo que las medusas puedan desarrollarse de una manera más rápida. Del mismo modo, la construcción de estructuras costeras, proporcionan lugares donde pueden proliferar con mayor facilidad.

%Teniendo en cuenta todo esto, se ha observado en estudios que las fluctuaciones ambientales aleatorias tienen poca influencia en el desarrollo de estas colonias mientras que los procesos deterministas cobran un importancia mucho mayor en comparación. Esto enfatiza la importancia de un análisis de las condiciones ambientales que provocan estos brotes para lograr anticiparlos.\cite{art:ArticuloCanepa_1}

Teniendo en cuenta todo esto,diferentes estudios concluyen que estas alteraciones aleatorias del medio, tiene poca influencia en el desarrollo de las colonias de medusas, mientras que las condiciones ambientales que se repiten anualmente tiene una mayor importancia en comparación. Esto remarca la importancia de un análisis de las condiciones ambientales que provocan estos brotes para poder anticiparse a ellos.\cite{art:ArticuloCanepa_1}

\section{Knowledge Discovery in Databases (KDD)}
\textcolor{red}{Explicación}
\subsection{Preprocesamiento de los datos}
Los datos obtenidos inicialmente en bruto no pueden ser utilizados directamente para la minería de datos. Estos datos deben ser preprocesados para eliminar posibles variables que no son necesarias o datos incorrectos y así conseguir transformar nuestro conjunto de datos iniciales, en un conjunto más útil y menos pesado por lo que el algoritmo de análisis posterior, tendrá una menor carga de trabajo. 
\textcolor{red}{Reescribir con más puntos}
\subsection{Minería de datos}
Actualmente se recopila una gran cantidad de información de todos los ámbitos y es necesario darla un uso práctico. La \textbf{minería de datos} es un campo de las \textcolor{red}{ciencias de la computación por el cual se tratan de descubrir nuevos patrones o relaciones en conjuntos de datos y así, conseguir un conocimiento obtenido de manera automática (\emph{Machine Learning}}). Con estas nuevas relaciones se trata de explicar comportamientos actuales o predecir resultados futuros~ \cite{mineria_tecnicas_herramientas}. 

Sin embargo, la minería de datos es solo una fase del proceso de descubrimiento del conocimiento (KDD) pues es necesario tratar los datos antes de analizarlos así como validarlos posteriormente. Este proceso podemos obtener las siguientes fases:

\imagenflotante{esquemaMineria.png}{Fases del proceso de KDD~\cite{mineria_tecnicas_herramientas}.}

%
%
%Mirar correccion
%
%

\subsection{Machine learning}







%Algunos conceptos teóricos de \LaTeX \footnote{Créditos a los proyectos de Álvaro López Cantero: Configurador de Presupuestos y Roberto Izquierdo Amo: PLQuiz}.
%
%
%
%\section{Secciones}
%
%Las secciones se incluyen con el comando section.
%
%\subsection{Subsecciones}
%
%Además de secciones tenemos subsecciones.
%
%\subsubsection{Subsubsecciones}
%
%Y subsecciones. 
%
%
%\section{Referencias}
%
%Las referencias se incluyen en el texto usando cite \cite{wiki:latex}. Para citar webs, artículos o libros \cite{koza92}.
%
%
%\section{Imágenes}
%
%Se pueden incluir imágenes con los comandos standard de \LaTeX, pero esta plantilla dispone de comandos propios como por ejemplo el siguiente:
%
%\imagen{escudoInfor}{Autómata para una expresión vacía}
%
%
%
%\section{Listas de items}
%
%Existen tres posibilidades:
%
%\begin{itemize}
%	\item primer item.
%	\item segundo item.
%\end{itemize}
%
%\begin{enumerate}
%	\item primer item.
%	\item segundo item.
%\end{enumerate}
%
%\begin{description}
%	\item[Primer item] más información sobre el primer item.
%	\item[Segundo item] más información sobre el segundo item.
%\end{description}
%	
%\begin{itemize}
%\item 
%\end{itemize}
%
%\section{Tablas}
%
%Igualmente se pueden usar los comandos específicos de \LaTeX o bien usar alguno de los comandos de la plantilla.
%
%\tablaSmall{Herramientas y tecnologías utilizadas en cada parte del proyecto}{l c c c c}{herramientasportipodeuso}
%{ \multicolumn{1}{l}{Herramientas} & App AngularJS & API REST & BD & Memoria \\}{ 
%HTML5 & X & & &\\
%CSS3 & X & & &\\
%BOOTSTRAP & X & & &\\
%JavaScript & X & & &\\
%AngularJS & X & & &\\
%Bower & X & & &\\
%PHP & & X & &\\
%Karma + Jasmine & X & & &\\
%Slim framework & & X & &\\
%Idiorm & & X & &\\
%Composer & & X & &\\
%JSON & X & X & &\\
%PhpStorm & X & X & &\\
%MySQL & & & X &\\
%PhpMyAdmin & & & X &\\
%Git + BitBucket & X & X & X & X\\
%Mik\TeX{} & & & & X\\
%\TeX{}Maker & & & & X\\
%Astah & & & & X\\
%Balsamiq Mockups & X & & &\\
%VersionOne & X & X & X & X\\
%} 
