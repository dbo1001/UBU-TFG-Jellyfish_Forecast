\capitulo{3}{Conceptos teóricos}

\begin{comment}
En aquellos proyectos que necesiten para su comprensión y desarrollo de unos conceptos teóricos de una determinada materia o de un determinado dominio de conocimiento, debe existir un apartado que sintetice dichos conceptos.
\end{comment}



\section{Minería de datos}
Actualmente se recopila una gran cantidad de información de todos los ámbitos y es necesario darla un uso práctico. La \textbf{minería de datos} es un campo de la ciencia de la computación por el cual se tratan de descubrir nuevos patrones o relaciones en conjuntos de datos y así, conseguir un conocimiento obtenido de manera automática (\emph{Machine Learning}). Con estas nuevas relaciones se trata de explicar comportamientos actuales o predecir resultados futuros. \cite{mineria_tecnicas_herramientas}

Sin embargo, la minería de datos es solo una fase del proceso de descubrimiento del conocimiento (KDD) pues es necesario tratar los datos antes de analizarlos así como validarlos posteriormente. Este proceso podemos obtener las siguientes fases:

\imagen{esquemaMineria.png}{Fases del proceso de minería de datos.\cite{mineria_tecnicas_herramientas}}

\subsection{Preprocesamiento de los datos}
Los datos obtenidos inicialmente no pueden ser utilizados directamente para la minería de datos. Estos datos deben ser preprocesados para eliminar posibles variables que no son necesarias o datos incorrectos y así conseguir transformar nuestro conjunto de datos iniciales en un conjunto más útil y menos pesado por lo que el algoritmo de análisis posterior, tendrá una menor carga de trabajo. 

\subsection{Machine learning}






\begin{comment}
Algunos conceptos teóricos de \LaTeX \footnote{Créditos a los proyectos de Álvaro López Cantero: Configurador de Presupuestos y Roberto Izquierdo Amo: PLQuiz}.



\section{Secciones}

Las secciones se incluyen con el comando section.

\subsection{Subsecciones}

Además de secciones tenemos subsecciones.

\subsubsection{Subsubsecciones}

Y subsecciones. 


\section{Referencias}

Las referencias se incluyen en el texto usando cite \cite{wiki:latex}. Para citar webs, artículos o libros \cite{koza92}.


\section{Imágenes}

Se pueden incluir imágenes con los comandos standard de \LaTeX, pero esta plantilla dispone de comandos propios como por ejemplo el siguiente:

\imagen{escudoInfor}{Autómata para una expresión vacía}



\section{Listas de items}

Existen tres posibilidades:

\begin{itemize}
	\item primer item.
	\item segundo item.
\end{itemize}

\begin{enumerate}
	\item primer item.
	\item segundo item.
\end{enumerate}

\begin{description}
	\item[Primer item] más información sobre el primer item.
	\item[Segundo item] más información sobre el segundo item.
\end{description}
	
\begin{itemize}
\item 
\end{itemize}

\section{Tablas}

Igualmente se pueden usar los comandos específicos de \LaTeX o bien usar alguno de los comandos de la plantilla.

\tablaSmall{Herramientas y tecnologías utilizadas en cada parte del proyecto}{l c c c c}{herramientasportipodeuso}
{ \multicolumn{1}{l}{Herramientas} & App AngularJS & API REST & BD & Memoria \\}{ 
HTML5 & X & & &\\
CSS3 & X & & &\\
BOOTSTRAP & X & & &\\
JavaScript & X & & &\\
AngularJS & X & & &\\
Bower & X & & &\\
PHP & & X & &\\
Karma + Jasmine & X & & &\\
Slim framework & & X & &\\
Idiorm & & X & &\\
Composer & & X & &\\
JSON & X & X & &\\
PhpStorm & X & X & &\\
MySQL & & & X &\\
PhpMyAdmin & & & X &\\
Git + BitBucket & X & X & X & X\\
Mik\TeX{} & & & & X\\
\TeX{}Maker & & & & X\\
Astah & & & & X\\
Balsamiq Mockups & X & & &\\
VersionOne & X & X & X & X\\
} 
\end{comment}