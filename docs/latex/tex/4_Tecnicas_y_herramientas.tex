\capitulo{4}{Técnicas y herramientas}

Este apartado se plasman las diferentes técnicas metodológicas y herramientas de desarrollo que se han utilizado en la realización del proyecto, así como las posibles alternativas que se han tenido en cuenta y el motivo de haberlas desechado.

\section{Gestión del proyecto}\label{GesProyecto}
\subsection{Scrum}\label{Scrum}
Scrum se trata de un marco de trabajo ágil destinado al manejo de proyectos de desarrollo \emph{software}. Está destinado para equipos pequeños dividiendo el trabajo en objetivos que se van logrando de manera incremental a través de iteraciones denominadas \emph{sprints}~\cite{wiki:scrm}.

\subsection{Git}\label{Git}
Git se trata de un sistema de control de versiones gratuito y de código abierto para el manejo de proyecto. Se trata de un software de control de versiones de manera que se puede llevar un registro de cambios en los archivos y posibilita la coordinación de trabajos entre diferentes personas~\cite{git_1,git_2}.

\subsection{GitHub}\label{GitHub}
GitHub es un plataforma destinada a alojar proyectos, qu se basa en el software de control de versiones Git. Esta plataforma utiliza un interfaz web desde la que se nos permite realizar control de código, documentación, gestión de tareas y otros muchas funcionalidades además de integración con otros servicios. GitHub es gratuito para proyectos \emph{open source}~\cite{wiki:github_wiki,github}.

\subsubsection{Alternativas}\label{AlternativasGitHub}
Otras alternativas a GitHub fueron, Gitlab y Bitbucket. Ambos servicios son bastante similares a GitHub en funcionalidades y basados en Git.
\begin{itemize}
	\item Bitbucket fue rechazado rápidamente por la falta de familiaridad en su uso ya que no lo había usado nunca, únicamente había visto repositorios en la web.
	
	\item Gitlab es un entorno más conocido ya que es el software que he utilizado en las prácticas de empresa para el control de código dentro del equipo de trabajo del que formo parte.
\end{itemize}


Finalmente se decidió usar GitHub por haber sido utilizado en clase de gestión de proyectos por lo que se tenia un mayor conocimiento de su funcionamiento, así como por su integración con ZenHub del que se hablará a continuación.

\subsection{ZenHub}\label{ZenHub}
ZenHub es una herramienta para gestión de proyectos que se integra con GitHub. Este complemento añade a GitHub un tablero canvas en el cual se representan las \emph{issues}. Es posible estimar tareas, así como darlas prioridades y visualizar gráficos como el gráfico \emph{burndown}~ \cite{zenhub}.

\subsubsection{Alternativas}\label{AlternativasZotero}
Se plantearon otras alternativas como Trello o GitHub projects, pero finalmente se eligió  ZenHub por su facilidad de uso y el haber sido usado anteriormente, requisito que las otras alternativas no cumplían.

\section{Herramientas}

	\subsection{Python}\label{Python}
Para le desarrollo en Python se eligió Visual Studio Code. Esta es una aplicación totalmente gratuita basada en el framework Electron y posee gran cantidad de extensiones para facilitar la tarea a la hora de la programación como auto completado con IntelliSense. Además tiene integración con Git para el control de versiones \cite{wiki:vscode_wiki,vscode}.
		\subsubsection{Alternativas}\label{AlternativasIDE}
Se estudiaron otras alternativas como PyCharm o Jupyter Notebook pero finalmente se eligió Visual Studio Code por la familiaridad con la misma y gran compatibilidad con diferentes lenguajes.

	\subsection{Jupyter Notebook}\label{jupyterNotebook}
También se ha utilizado Jupyter Noteebook para el desarrollo con Python. Esta herramienta se ha utilizado para realizar todas las pruebas debido a su facilidad para documentar el código y la segmentación del mismo en apartados separados. 

	\subsection{tmux}\label{tmux}
Se trata de un herramienta que permite lanzar múltiples terminales en una misma ventana, cada uno de manera independiente y corriendo en segundo plano. Esta herramienta se ha utilizado para la descarga de los datos obtenidos de \emph{Copernicus}. En un principio se utilizó unicamente una conexión ssh con un equipo de cómputo de Universidad. Esto daba varios errores, por una parte, la conexión se cerraba perdiendo el proceso de descarga. También era necesario que un equipo personal estuviese encendido constantemente durante la descarga para que no se detuviera la conexión. Para solucionar estos problemas se abrió una sesión de tmux corriendo en segundo plano, pudiendo así cerrar a conexión ssh y abrirla para consultar el proceso.\\
\textcolor{red}{Añadir si se usa para la ejecucion del modelo.}

	\subsection{Pencil}
Software gratuito para la realización de prototipos de interfaces gráficas. Permite la instalación de paquetes para crear maquetas de múltiples plataformas. 
	
	\subsection{Creately}
Aplicación web utilizada en la fase de modelado para la creación de los diagramas de uso. Dispone de una versión gratuita con opciones reducidas.

\section{Documentación}\label{Documentación}
	\subsection{\LaTeX}\label{LaTex}
\LaTeX{} es un sistema de composición de texto plano destinado a la composición de textos con una calidad tipográfica alta. Incluye características diseñadas para la elaboración de documentación técnica y científica siendo un estándar en la publicación de documentos de investigación. \LaTeX{} es totalmente gratuito \cite{latex}.
	\subsubsection{Alternativas}\label{AlternativasLatex}
Otras opciones valoradas fueron Microsoft Word y OpenOffice. La Universidad proporciona una plantilla para \LaTeX{} y OpenOffice por lo que Microsoft Word fue la primera descartada. Finalmente debido al enfoque más técnico que proporciona \LaTeX{} frente a OpenOffice, se decidió utilizarlo.
	\subsection{\TeX studio}\label{Texmaker}
\TeX studio{} se trata de un editor de textos gratuito, que ofrece diversas herramientas para elaborar documentación con \LaTeX{} haciendo la escritura más confortable e intuitiva \cite{texstudio}.
	\subsubsection{Alternativas}\label{AlternativasTexmaker}
Las alternativas a este editor que se estudiaron fueron \TeX maker y Overleaf. Se terminó eligiendo \TeX studio por ser un editor instalado en local, permitiendo el trabajo aunque no se tuviera conexión a Internet así como por ser más intuitivo y fácil de utilizar que \TeX maker.\\

	\subsection{Zotero}\label{zotero}
La herramienta Zotero es un software gratuito para la gestión de referencias pudiendo recoger, organizar y citar creando referencias bibliográficas para cualquier editor. También cuenta con integración en le navegador. Una vez recopilados todas las citas, se puede exportar a un fichero Bib\TeX{} para la utilizarse con \LaTeX{}~\cite{zotero}.

\section{Bibliotecas}\label{librerias}
	\subsection{Flask}\label{Flask}
Flask es un framework ligero para el desarrollo de aplicaciones web en Python bajo el modelo Modelo-Vista-Controlador (MVC). Está diseñado para desarrollar aplicaciones de manera rápida y sencilla y con capacidad de escalar a aplicaciones más complejas \cite{flask}.
	\subsection{Xarray}\label{xarray}
Xarray se trata de un proyecto de código abierto desarrollado para Python que facilita el trabajo con matrices multidimensionales etiquetadas utilizando la librería NumPy. Consta de una gran variedad de funciones para el análisis y la visualización de estructuras de datos. Está inspirado en el funcionamiento de la librería pandas y diseñado para funcionar con archivos de tipo netCDF \cite{xarray}.	
	\subsection{Pandas}\label{pandas}
Esta biblioteca se trata de una extension de NumPy y está destinada a la manipulación y análisis de datos en lenguaje Python. Permite trabajar con estructuras de datos y operaciones para su trasformación pudiendo estas ser tablas temporales o series numéricas \cite{pandas}.
	\subsection{FtpLib}\label{FtpLib}
Biblioteca destinada a la implementación de la parte del cliente en el protocolo FTP. Desarrollada para el lenguaje Python, nos permite automatizar accesos a servidores FTP \cite{ftp_lib}.
	\subsection{tqdm}\label{tqdm}
Pequeña librería utilizada para mostrar una barra de progreso a la hora de realizar la descarga de los datos del FTP.
	\subsection{Folium}
Se trata de una biblioteca que nos permite la visualización de mapas, pudiendo superponer elementos en los mismos. Folium utiliza a su vez una biblioteca de javaScript llamada \emph{leaflet}.

	\section{Bootstrap}
Bootstrap es un conjunto de herramientas para facilitar el desarrollo en HTML, CSS y JavaScript.
