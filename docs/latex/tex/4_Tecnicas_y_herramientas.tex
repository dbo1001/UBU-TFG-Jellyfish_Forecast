\capitulo{4}{Técnicas y herramientas}

\begin{comment}
Esta parte de la memoria tiene como objetivo presentar las técnicas metodológicas y las herramientas de desarrollo que se han utilizado para llevar a cabo el proyecto. Si se han estudiado diferentes alternativas de metodologías, herramientas, bibliotecas se puede hacer un resumen de los aspectos más destacados de cada alternativa, incluyendo comparativas entre las distintas opciones y una justificación de las elecciones realizadas. 
No se pretende que este apartado se convierta en un capítulo de un libro dedicado a cada una de las alternativas, sino comentar los aspectos más destacados de cada opción, con un repaso somero a los fundamentos esenciales y referencias bibliográficas para que el lector pueda ampliar su conocimiento sobre el tema.
\end{comment}

\begin{comment}
scrum
git
github,gitlab,bitbucket
zenhub
IDE:vs code, pycharm, editor latex
latex,editor latex
zotero

lenguaje python,librerias
\end{comment}

Este apartado se plasman las diferentes técnicas metodológicas y herramientas de desarrollo que se han utilizado en la realización del proyecto así como las posibles alternativas que se han tenido en cuenta y el motivo de haberlas desechado.

\section{Scrum}\label{Scrum}

\section{Git}\label{Git}

\section{GitHub}\label{GitHub}

\section{ZenHub}\label{ZenHub}

