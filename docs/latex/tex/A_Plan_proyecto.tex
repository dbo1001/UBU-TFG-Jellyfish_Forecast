\apendice{Plan de Proyecto Software}

\section{Introducción}

En este apartado se procederá a explicar con detalle cual ha sido el resultado de la planificación del proyecto.

Está planificación se ha realizado utilizando  una metodología ágil basada en \emph{sprints} de una duración de una o dos semanas en función de las necesidades y el tiempo disponible debido a otras cargas de trabajo diferentes a este proyecto.

En estos \emph{sprints} se van marcando ciertos objetivos que serán revisados junto a los tutores en las reuniones al final de los mismos. Los objetivos del siguiente \emph{sprint} serán marcados durante dichas reuniones.
 
Para el control de tiempos se ha utilizado la herramienta ZenHub siendo la valoración de los \emph{Story Points} la siguiente:

\tablaSmall{Equivalencias \emph{Story Points} y tiempo estimado}{c c }{StoryPoints/tiempo}
{ \multicolumn{1}{l}{Story Points} & Estimación temporal \\}{ 
1            & 1 hora              \\ 
2            & 1,5 horas           \\ 
3            & 2 horas             \\ 
4            & 2,5 horas           \\ 
5            & 3 horas             \\ 
6            & 3,5 horas           \\ 
7            & 4 horas             \\ 
8            & 6 horas             \\ 
9            & 8 horas             \\ 
}

Aclarar que los gráficos \emph{Burn Down} de los primeros \emph{sprints} no están todo lo bien que deberían por la poca experiencia con la herramienta.
 
\textcolor{red}{tabla en su sitio y etiquetas}

\section{Planificación temporal}

\subsection{Sprint 1 (29/01/2020 - 05/02/2020)}\label{Sprint-1}
En esta primera reunión se marcó el comienzo del proyecto. Ya se había hablado anteriormente con uno de los tutores (Jose Francisco) del interés sobre el proyecto propuesto del que también formaba parte de los tutores Álvar Arnaiz.

Al ser la primera reunión se hablo de las herramientas que se iban a utilizar así como acordar los primeros objetivos de este \emph{sprint}:

\begin{itemize}
\item Crear el repositorio.
\item Añadir la plantilla de Latex a la documentación.
\item Crear cuenta en la plataforma \emph{Copernicus}.
\item Investigar el funcionamiento básico de las librerías a utilizar.
\item Leer una serie de papers que me proporcionaron sobre las medusas.
\end{itemize}

Las \emph{issues} para este \emph{Sprint} se pueden ver \href{https://github.com/psnti/TFG-Pablo-Santidrian-Tudanca/milestone/1}{aquí}.

Se estimó unas 10 horas de trabajo de las que finalmente se invirtieron 8 horas quedando sin terminar una \emph{issue}.

\imagen{Sprint1_BurnDown.png}{\textit{Burndown chart} del Sprint 1}

\subsection{Sprint 2 (13/02/2020 - 28/02/2020)}\label{Sprint-2}

En la segunda reunión se comentó la existencia de una API para la descarga de los datos meteorológicos como alternativa a la descarga de una gran cantidad de datos a través del FTP.

Por otro lado, se me proporcionó apuntes de la asignatura de minería de datos para su lectura y aprendizaje.

Por último, los tutores me recomendaron iniciar la documentación del plan de proyecto de los \emph{Sprints} que se fuesen sucediendo para no acumular trabajo y se pudiera olvidar detalles del mismos.

Los objetivos fueron los siguientes:

\begin{itemize}
\item Realizar script para la descarga de los datos.
\item Comenzar a documentar el plan del proyecto.
\item Lectura de apuntes y papers.
\end{itemize}

Las \emph{issues} para este \emph{Sprint} se pueden ver \href{https://github.com/psnti/TFG-Pablo-Santidrian-Tudanca/milestone/2}{aquí}.

Se estimaron unas 8 horas de trabajo de las que finalmente se invirtieron 9 horas.

\imagen{Sprint2_BurnDown.png}{\textit{Burndown chart} del Sprint 2}

\subsection{Sprint 3 (28/02/2020 - 17/03/2020)}\label{Sprint-3}

En esta tercera reunión hablamos sobre la descarga de los datos, de las dos opciones posibles nos quedamos con la descarga por FTP por ser más fiable. Además este \emph{Sprint} se centrará en su mayor parte en documentación como las herramientas a utilizar o cuestiones teóricas sobre las medusas aparte de comenzar a desarrollar la base de la aplicación web.

Se marcaron los siguientes objetivos:
\begin{itemize}
\item Comienzo desarrollo web.
\item Elaboración de parte de la documentación teórica.
\item Descarga de los datos en un equipo de computo habilitado en la universidad.
\end{itemize}

Las \emph{issues} para este \emph{Sprint} se pueden ver \href{https://github.com/psnti/TFG-Pablo-Santidrian-Tudanca/milestone/3}{aquí}.

Se estimó unas 19 horas y finalmente se realizaron 24. La causa del desvió de horas principalmente fueron, el comienzo del desarrollo web por el desconocimiento previo y el comentar el código creado para la descarga de los datos necesarios pues se corrigieron errores y se mejoró la salida por pantalla con una barra de descarga más visual.


\imagen{Sprint3_BurnDown.png}{\textit{Burndown chart} del Sprint 3}

\subsection{Sprint 4 (17/03/2020 - 30/03/2020)}\label{Sprint-3}

La cuarta reunión se hizo mediante \emph{Skype} con Jóse Francisco debido a la cuarentena por el coronavirus. Se habló sobre la necesidad de utilizar la VPN de la universidad por este mismo motivo para poder tener acceso a la maquina remota. 

Por otra parte, se mostró el avance de la web acordando que el siguiente paso debería ser la introducción de los mapas para lo que se comentaron varias bibliotecas de las que se podía hacer uso.

Los objetivos que se marcaron fueron:
\begin{itemize}
	\item Continuación desarrollo web.
	\item Introducción de mapas en la aplicación web.
	\item Conexión a la VPN de la universidad para conseguir dejar los datos descargándose aunque la sesión esté cerrada.
	\item Continuación de la documentación.
\end{itemize}

Las \emph{issues} para este \emph{Sprint} se pueden ver \href{https://github.com/psnti/TFG-Pablo-Santidrian-Tudanca/milestone/4}{aquí}.

\imagen{Sprint4_BurnDown.png}{\textit{Burndown chart} del Sprint 4}

Se estimaron 15 hora y media y finalmente se realizaron 16.

\subsection{Sprint 5 (30/03/2020 - \textcolor{red}{RELLENAR})}

Esta reunió se realizó también de manera remota por \emph{Skype} con ambos tutores. Se mostró la implementación de los mapas en la aplicación web así como varias mejoras en la interfaz de la misma. También avances realizados en la memoria y ciertas mejoras propuestas por los tutores.

Sobre la web, a pesar de haber empezado el desarrollo de la web, se recomendó el realizar unos bocetos de la misma para definir la estructura a seguir.

Por otro lado, una vez descargados los datos oceánicos el siguiente paso será generar una estructura de datos para poder entrenar al modelo.

Por último, se acordó que el siguiente paso en la realización de la memoria debía ser la finalización de los objetivos del proyecto y definir los requisitos.



Los objetivos que se marcaron fueron los siguientes:
\begin{itemize}
	\item Generar estructura de datos.
	\item Definir objetivos del proyecto.
	\item Definir requisitos.
	\item Definir aspectos relevantes.
%	\item Definir casos de uso.
	\item Realizar bocetos aplicación web.
\end{itemize}

Las \emph{issues} para este \emph{Sprint} se pueden ver \href{https://github.com/psnti/TFG-Pablo-Santidrian-Tudanca/milestone/5}{aquí}.

\textcolor{red}{IMAGEN GRAFICO BURNDOWN}
%\imagen{Sprint4_BurnDown.png}{\textit{Burndown chart} del Sprint 4}

Se estimaron 13 \textcolor{red}{revisar} horas y finalmente se realizaron XXX.

\section{Estudio de viabilidad}

\subsection{Viabilidad económica}

\subsection{Viabilidad legal}


