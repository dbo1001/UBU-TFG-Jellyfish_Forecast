\capitulo{1}{Introducción}

\begin{comment}
Descripción del contenido del trabajo y del estrucutra de la memoria y del resto de materiales entregados.
\end{comment}

El cambio climático está provocando que los fenómenos naturales extremos sean cada vez más habituales. Esto afecta a distintas poblaciones locales como pueden ser la de las medusas.

Las medusas tienen periodos de aparición estacionales y se alimentan de plancton, por lo que su densidad es mayor en zonas donde este abunda. Estas zonas suelen ser lugares cercanos al talud continental donde además se reproducen \cite{noauthor_proliferaciones_nodate}. %Las medusas son organismos asexuales, un mismo individuo puede generar descendencia y esta se lleva a cabo mediante la fecundación de gametos que se convertirán en larvas, más adelante estas se adherirán a alguna superficie donde se transformarán en pólipos para finalmente se separará la medusa adulta~\cite{noauthor_reproduccion_2016}.

La aparición de medusas cerca de las costas, es un fenómeno que se da cada vez con mayor frecuencia. Estas floraciones tiene efectos perjudiciales en ámbitos como el turismo o la pesca, así como los daños que pueden provocar a la salud de las personas llegando en algunos casos a causar enfermedades graves \cite{art:picaduras_1,art:picaduras_2}. 

Alguno de los factores que están provocando el aumento de los acercamientos de las medusas a las playas son~\cite{noauthor_proliferaciones_nodate,art:ArticuloCanepa_1}:
\begin{itemize}
	\item La \textbf{climatología}, influye principalmente el cambio climático con el descenso del nivel de lluvias y el aumento de las temperaturas, que favorecen el aumento de la salinidad y de la temperatura del agua. 
	\item La \textbf{contaminación} provocada por la modificación de las zonas costeras o los vertidos cercanos a los costas provocan la proliferación de bacterias o plancton que sirve de alimento para las medusas.
	\item La \textbf{sobrepesca} causa un descenso de depredadores así como de otras especies con las que las medusas compiten por el alimento.
\end{itemize}

Con este proyecto se pretende generar un modelo con el que predecir el comportamiento de las poblaciones de medusas en las costas de Chile en función de datos meteorológicos y marítimos obtenidos a través del programa europeo \emph{Copernicus}\footnote{Programa de observación terrestre de la Unión Europea. \url{https://marine.copernicus.eu/}}. 

La zona de estudio del proyecto se centra en Chile, pues son los datos de los que se dispone gracias a los colaboradores de la Universidad.

Un modelo es el resultado de aplicar un algoritmo a un conjunto de datos. El producto obtenido, es un conjunto reglas y patrones que pueden ser aplicados a otro grupos de datos con los que conseguir predicciones~\cite{modelo_definicion}.

Para conseguir este modelo se van a aplicar técnicas de minería de datos y \emph{machine learning}. Estas, se pueden entender como procesos para el descubrimiento de patrones y relaciones entre los diferentes datos de manera que nos permitan realizar predicciones a partir de los mismos.

Con el objetivo de poder aplicar estos algoritmos de la manera más eficaz posible, los datos son pre-procesados, es decir se eliminará la información (variables) que no es útil y se delimitará la zona geográfica de estudio. 


