\capitulo{1}{Introducción}

\begin{comment}
Descripción del contenido del trabajo y del estrucutra de la memoria y del resto de materiales entregados.
\end{comment}

El cambio climático está provocando que los fenómenos naturales extremos sean cada vez más habituales afectando a la poblaciones locales. Una de estas poblaciones locales son la medusas.

Las medusas tienen periodos de aparición estacionales y se alimentan de plancton, por lo que su densidad es mayor en zonas donde este abunda. Estas zonas suelen ser lugares cercanos al talud continental donde además se reproducen \cite{noauthor_proliferaciones_nodate}. %Las medusas son organismos asexuales, un mismo individuo puede generar descendencia y esta se lleva a cabo mediante la fecundación de gametos que se convertirán en larvas, más adelante estas se adherirán a alguna superficie donde se transformarán en pólipos para finalmente se separará la medusa adulta~\cite{noauthor_reproduccion_2016}.

La aparición de medusas cerca de las costas, es un fenómeno que se da cada vez con mayor frecuencia. Estas floraciones tiene efectos perjudiciales en ámbitos como el turismo o la pesca, así como los daños que pueden provocar a la salud de las personas llegando en algunos casos a causar enfermedades graves \cite{art:picaduras_1,art:picaduras_2}. 

Alguno de los factores que están provocando el aumento de los acercamientos de las medusas a las playas son~\cite{noauthor_proliferaciones_nodate,art:ArticuloCanepa_1}:
\begin{itemize}
	\item La \textbf{climatología}, influye principalmente el cambio climático con el descenso del nivel de lluvias y el aumento de las temperaturas, que favorecen el aumento de la salinidad y de la temperatura del agua. 
	\item La \textbf{contaminación} provocada por la modificación de las zonas costeras o los vertidos cercanos a los costas provocan la proliferación de bacterias o plancton que sirve de alimento para las medusas.
	\item La \textbf{sobrepesca} causa un descenso de depredadores así como de otras especies con las que las medusas compiten por el alimento.
\end{itemize}

Con este proyecto se pretende predecir el comportamiento de las poblaciones de medusas en las costas de Chile en función de datos meteorológicos y marítimos obtenidos mediante el programa europeo \emph{Copernicus}\footnote{Programa de observación terrestre de la Unión Europea. \url{https://marine.copernicus.eu/}}. Estos datos se preprocesarán, es decir se eliminará la información que no es útil y delimitar la zona geográfica de estudio. A partir de ahí se entrenará una serie de modelos para predecir la llegada a las costas de las medusas.\\
\textcolor{red}{Extender tema del modelo, que se eniende por modelo, machine learning...}