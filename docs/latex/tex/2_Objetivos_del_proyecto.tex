\capitulo{2}{Objetivos del proyecto}

%Este apartado explica de forma precisa y concisa cuales son los objetivos que se persiguen con la realización del proyecto. Se puede distinguir entre los objetivos marcados por los requisitos del software a construir y los objetivos de carácter técnico que plantea a la hora de llevar a la práctica el proyecto.

A continuación, se detallarán los objetivos que han motivado la realización de este proyecto.

\section{Objetivos generales}
\begin{itemize}
	
	\item Generar un modelo predictivo capaz de pronosticar la presencia de medusas.
	\item Utilizar técnicas de minería de datos.
	\item Recopilar y filtrar los datos necesarios para el modelo predictivo.
	\item Desarrollo de una aplicación web permitiendo la consulta de las predicciones a los usuarios.
\end{itemize}

\section{Objetivos técnicos}
\begin{itemize}
	\item Generar documentación en \LaTeX, en concreto, con la herramienta \TeX studio.
	\item Utilizar un sistema de control de versiones con la plataforma GitHub junto a la extensión ZenHub para facilitar la gestión del proyecto.
	\item Generar script para recopilar y filtrar los datos necesarios para la realización del proyecto.
	\item Generar una estructura de datos sobre la que se obtendrán los modelos, utilizando los datos de avistamientos y oceánicos obtenidos.
	\item Comparar los resultados de los diferentes modelos obtenidos.	
	\item Realizar una web en la que mostrar los resultados del modelo de una manera fácil e intuitiva.
\end{itemize}

\section{Objetivos personales}
\begin{itemize}
	\item Investigar diferentes técnicas y herramientas utilizadas para la minería de datos.
	\item Adquirir conocimientos sobre el desarrollo web.
	\item Aprender a generar documentación en \LaTeX.
	
\end{itemize}