\apendice{Documentación de usuario}

\section{Introducción}

%Este apéndice recoge todo lo que un usuario debe saber para poder ejecutar la aplicación. Cabe destacar que para un cualquier usuario no seria necesario la instalación de ninguna herramienta pues la aplicación está desplegada\footnote{Página web Jellyfish Forecast. \url{https://jellyfish-forecast.herokuapp.com/}} y puede visitarse desde cualquier navegador. A pesar de ello, se explicará como instalar y ejecutar la aplicación desde un servidor local.

Este apéndice recoge todo lo que un usuario debe saber para poder ejecutar la aplicación y los requisitos mínimos que se necesitan. 

\section{Requisitos de usuarios}
%Para la ejecución de la aplicación se deberá tener instalado Python (el desarrollo se ha realizado con la versión 3.7.6). Posteriormente se instalarán los paquetes recogidos en el fichero requirements.txt. La realización del proyecto ha sido en una máquina con sistema operativo W10, pero debería poder ejecutarse en otros sistemas sin problemas.
La aplicación se encuentra desplegada por lo que la única instalación necesaria seria la de un navegador web. En el apartado de pruebas de la aplicación se recogen algunos navegadores que han sido probados y funcionan correctamente \ref{pruebas_navegadores}.

%\section{Instalación}
%
%Para la instalación de el proyecto se deberán seguir los siguientes pasos. Estos pasos son los mismos que los recogidos en el apéndice anterior \ref{Descarga} con un menor detalle:
%
%\begin{enumerate}
%	\item En primer lugar se ha de acceder al repositorio\footnote{Web Jellyfish Forecast. \url{https://github.com/psnti/WebJellyfishForecast}}.
%	\item Descargar el contenido desde \textbf{Clone or Download}
%	\imagen{clone_download.png}{Descargar repositorio}\label{clone}.
%	\item Descomprimir el fichero .zip en la ruta deseada.
%	\item Antes de instalar las bibliotecas utilizadas en el proyecto crearemos un entorno virtual y accederemos a el:
%	\begin{verbatim}
%	python -m venv env
%	env\Scripts\active.bat
%	\end{verbatim}
%	\item Para la instalación de la biblioteca necesarias, contamos con el archivo requirements.txt. Ejecutando el siguiente comando, se instalarán todas las dependencias del proyecto en  nuestro entorno virtual.
%	\begin{verbatim}
%	pip install -r requirements.txt
%	\end{verbatim}
%\end{enumerate}

\section{Manual del usuario}

Una vez dentro de la web ya sea ejecutándolo en un servidor local, o desde
pagina desplegada,las acciones se realizan de la misma manera.

En primer lugar nos encontramos con una página de inicio meramente informativa.

\imagen{pagina_inicio_web.png}{Página de inicio}\label{pagina_inicio}

Desde ahí, podremos acceder a la pestaña de <<Mapas>> donde se encuentra la funcionalidad de la aplicación.

\imagen{pagina_mapas_web.png}{Página de consulta}\label{pagina_mapas_web}

Para realizar una predicción, deberemos introducir una fecha y una playa. En caso de seleccionar unicamente una playa sin fecha, aparecerá un mensaje de error indicándonoslo.

\imagen{pagina_error_fecha.png}{Error en la consulta}\label{pagina_error_fecha}

Una vez seleccionadas ambas opciones correctamente, al hacer click en <<Consultar>> nos aparecerán dos gráficos y el mapa realizará un zoom a la playa seleccionada. Estos gráficos se tratan, de la predicción realizada y el historial de avistamientos de la playa seleccionada. 

\imagen{pagina_consulta.png}{Página de consulta con gráficos}\label{pagina_consulta}

Una vez realizada la predicción, existe la opción de exportar dicha predicción a un fichero de tipo .xlsx

\textcolor{red}{imagen fichero exportado}

Por último, nos encontramos la pestaña de contacto donde se sitúan enlaces a la página de la Universidad, el repositorio y enlaces de contacto.

\imagen{pagina_contacto.png}{Página de contacto}\label{pagina_contacto}