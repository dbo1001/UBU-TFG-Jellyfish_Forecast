\apendice{Especificación de Requisitos}

\section{Introducción}
En este apéndice se recogerán los diferentes objetivos del proyecto con sus correspondientes requisitos funcionales y no funcionales que marcan el desarrollo de este proyecto.

\section{Objetivos generales}

Este proyecto tiene como objetivo el desarrollo de un modelo que nos permita predecir la presencia de medusas en las costas de Chile en funciones de las condiciones marítimas.\\
El modelo resultante se implementará en un aplicación web con la que poder consultar la predicción de la fecha requerida ayudando a su visualización mediante representaciones gráficas.

\section{Catálogo de requisitos}

	\subsection{Requisitos funcionales}

\begin{description}
	\item[RF-1 Obtención de los datos:] Se debe ser capaz de descargar los datos necesarios de manera automática a través de un servidor FTP.
	\item[RF-2 Filtrado de los datos:] Los datos descargados han de ser tratados descartando las zonas geográficas distintas al lugar de estudio así como las variables ambientales que no sean de utilidad.
	\item[RF-3 Cruce de datos:] Los datos filtrados se han de cruzar con los de avistamientos, obteniendo una estructura de los avistamientos con su respectiva fecha, localización y variables marítimas.
	
	\item[RF-X Mostrar mapa:] La aplicación debe ser capaz de mostrar un mapa con el que poder interactuar.
	\item[RF-X Elección de fechas:] Se debe poder seleccionar una fecha de la que obtener información.
	\item[RF-X Elección de playa:] Se debe poder elegir una playa de la que obtener información.
	\item[RF-X Visualzación de resultados:] El usuario debe ser capaz de visualizar los resultados de una playa en la fecha especificada.
	\item[RF-X] ¿Fechas valida? ¿nombre de las playa valido?
	
\end{description}

	\subsection{Requisitos no funcionales}

\begin{description}
	\item[RNF-1 Rendimiento:] La aplicación debe tener buenos tiempos de respuesta.
	\item[RNF-2 Usabilidad:] La aplicación debe ser intuitiva, de manera que al usuario no le suponga un esfuerzo su uso.
	\item[RNF-3 Diseño \emph{responsive}:] Se debe garantizar una correcta visualización en diferentes dispositivos de distintas dimensiones.
	\item[RNF-4 Internacionalización] La aplicación debe disponer de varios idiomas.
\end{description}


\section{Especificación de requisitos}
	\subsection{Actores}
Solo existe un tipo de actor, aquel usuario que consulta las predicciones.
	\subsection{Diagrama de casos de uso}

	\subsection{Casos de uso}
	
	\textcolor{red}{Casos de uso: Consultar predicción, Seleccionar una playa concreta.}

\tablaSmallSinColores{Caso de uso X: caso de uso X}{p{3cm} p{.75cm} p{9cm}}{tablaCUX}{
	\multicolumn{3}{p{10.25cm}}{CU-X: Caso de uso cualquiera} \\
}
{
	Descripción                            & \multicolumn{2}{p{10.25cm}}{Descripción} \\\cline{1-3}
	Precondiciones                         & \multicolumn{2}{p{10.25cm}}{Precondiciones} \\\cline{1-3}
	Requisitos                         	   & \multicolumn{2}{p{10.25cm}}{Requisitos} \\\cline{1-3}
	Usuario                         	   & \multicolumn{2}{p{10.25cm}}{Usuario} \\\cline{1-3}
	\multirow{3}{3.5cm}{Secuencia normal}  & Paso & Acción \\\cline{2-3}
	& 1    & Paso \\\cline{1-3}
	Excepciones							   & Excepciones \\\cline{1-3}
	Postcondiciones                        & \multicolumn{2}{p{10.25cm}}{Postcondiciones} \\\cline{1-3}
	Frecuencia                             & Frecuencia \\
	Importancia                            & Importancia \\
}
	

